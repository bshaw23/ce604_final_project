\section{Variational Equation}
The variational equation (a.k.a virtual work) is the next step towards FEM.
It is shown in~\cref{eq:virtual_work}.

\begin{equation}
\int_{\Omega} \sigma_{ij,j} \, \hat{u}_i \, d\Omega 
+ \int_{\Omega} f_i \, \hat{u}_i \, d\Omega 
= 0
\label{eq:virtual_work}
\end{equation}

By applying the divergence theorem, boundary conditions, and changes of variables, we arrive at~\cref{eq:fem}.

\begin{equation}
0 = \sum_{e=1}^{n_{el}} \int_{0}^{l^e} 
\left(
\hat{u}_1 \bar{u}_1 \ddot{u}_1 + 
\hat{u}_2 \bar{u}_2 \ddot{u}_2 + 
\hat{u}_3 \bar{u}_3 \ddot{u}_3 + 
\hat{\theta}_x \bar{I}_x \ddot{\theta}_x + 
\hat{\theta}_y \bar{I}_y \ddot{\theta}_y + 
\hat{\theta}_z \bar{I}_z \ddot{\theta}_z
\right) d\xi
\label{eq:fem}
\end{equation}

This is the discretized weak form of the virtual work equation, which is used for FEM formulation.
Essentially, this is the boundary between continuum mechanics and FEM.
To continue into FEM basis functions are applied, and the equations are implemented in a matrix formulation.
To continue to derive FEM for a Timoshenko beam refer to~\cite{hughes-fem}.