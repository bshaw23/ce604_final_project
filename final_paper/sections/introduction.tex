\section{Introduction}

%The purpose of this project is to make a clear connection between the fields of continuum mechanics and the finite element method. 
%The material presented herein is based on Hughes chpt 5.4 and Dr. Shepherd's Timoshenko Frame Derivation.
%The motivation for undertaking this project is first and foremost to help myself gain a better understanding of these topics. 
%I also hope that my work here will also help to accelerate other students learning as they try to understand the connection between continuum mechanics and the finite element method. 
%
%{\Rd make this different, more like a real paper}
%I think it would be helpful to say $\alpha$ and $\beta$ (greek) represent 1 and 2 while $e$ and $i$ (latin) represent 1, 2, and 3.

Obtaining an understanding of the bridge between theory and application of the Finite Element Method (FEM) can be difficult.
Even after taking classes in both application and theory, understaning  may still be limited.
Currently, the gold standard for understanding FEM is~\cite{hughes-fem}.
Although it is known to be a well written book, the threshold for understanding is high.
Dr. Kendrick Shepherd has taken steps to make this work more navigable for those early on in their understanding. 
He has produced a number of documents elaborating on the concepts of~\cite{hughes-fem}.
Yet, for many the theory behind FEM is vague.

This work has been compiled as a continuing effort to help make the theory behind FEM more accessible and will be applied to a beam.
Much of what is presented is an elaboration on what is found in chapter 5.4 of ~\cite{hughes-fem}.
This work also relies heavily on the derivation of the Timoshenko frame analysis~\cite{shepherd-frame}.

A note regarding notation used in this work.
Subscripts and indexing are important in the formulation of finite element analysis.
Following the example of ~\cite{hughes-fem}, the subscripts  $\alpha$ and $\beta$ will represent the integers 1 and 2.
Other latin letters (e.g. $i,j,k,l$) will typically represent the integers 1,2,3, with the expection of $e$ and $n$ being any arbitarity integer.

The work will proceed in the following manner.
First, the assumptions made in Timoshenko beam theory are explained.
Particular attention will be placed on justifying these assumptions. 
Next, the theory of linear elasticity as provided in continuum mechanics is put forth.
Finally, the variational equation (virtual work equation) will be stated, about which the previous sections provide background.
A breif explanation of how to proceed toward the FEM will conclude. 
